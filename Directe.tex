\documentclass[12pt,draft]{article}
\usepackage[couleur]{dipneuste}%ct%thm%couleur%draft%minimal

\geometry{a4paper, hmargin=.15 \paperwidth, vmargin=.12 \paperheight}

\setcounter{secnumdepth}{5}
\renewcommand{\theparagraph}{\bf\color{DarkBlue}(\arabic{section}.\arabic{subsection}.\alph{paragraph})}

\renewcommand{\Re}{\texttt{Re}}
\renewcommand{\Im}{\texttt{Im}}

\begin{document}
\section{Intégrabilité}
\subsection{Notations}
\paragraph*{Données.}
Soit $M$ une variété réelle lisse ou analytique de dimension $2n$ munie d'une structure complexe $I$ intégrable. On dispose au voisinage d'un point $O \in M$ de coordonnées réelles $x^i$ centrées en $O$. On notera $u^a = x^a + ix^{a+n}$	.

On introduit les objets suivants~:
\paragraph{} Les différents faisceaux naturels, (on notera $\Gamma(U,\Ff)$ les sections de $\Ff$ sur l'ouvert $U$)~:
\begin{itemize}
\item Les faisceaux constants $\underline{\R}$, $\underline{\C}$. Dont les sections sont les fonctions localement constantes à valeurs dans $\R$, resp. $\C$.
\item Le faisceau structural (lisse) $\Cc^\infty$ ou $\Cc^\infty_M$, dont les sections sont les fonctions lisses à valeurs complexes.
\item Le faisceau structural analytique $\Cc^\omega$ ou $\Cc^\omega_M$, dont les sections sont les fonctions analytiques à valeurs complexes.
\end{itemize}
\paragraph{} Le fibré tangent $T_\R M$ à $M$, c'est de manière naturelle un faisceau de $\R$-espaces vectoriels de dimensions $2n$. Une base locale des sections est donnée par la famille~:
\[
\dpp{}{x^1} , \dpp{}{x^2} , \cdots , \dpp{}{x^{2n}}
\]
On prend comme définition qu'un champ de vecteur est une dérivation réelle sur l'anneau des fonctions réelles sur $M$.
\paragraph{} Son complexifié $T_\C M := T_\R M \otimes_\R \C$ (fibre à fibre ou, ce qui revient au même le produit tensoriel au dessus de faisceau constant de corps $\underline{\R}$ avec le faisceau de $\underline{\R}$-espaces vectoriels $\underline{\C}$).
Une base locale des sections est donnée par la famille précédente ou par~:
\[
\dpp{}{u^1} , \dpp{}{u^2} , \cdots , \dpp{}{u^{n}} , \dpp{}{\bar{u}^1} , \dpp{}{\bar{u}^2} , \cdots , \dpp{}{\bar{u}^{n}}
\]
On a la décomposition spectrale (fibre à fibres ou en tant que faisceaux)
\[
T_\C M = T^{1,0}M \oplus T^{0,1}M
\]
qui fait apparaître
\begin{itemize}
\item L'espace propre pour la valeur propre $i$ de l'opérateur $I$~: $T^{1,0}M$. Fibré vectoriel complexe de dimension $n$ dont les sections sont appelés \textit{champs de vecteurs $I$-holomorphes}.\\
Il s'identifie naturellement au fibré tangent réel par l'application partie réelle $T^{1,0}M~\rightarrow~T_\R M$ ; on le notera par la suite $TM$.
\item L'espace propre pour la valeur propre $-i$ de l'opérateur $I$~: $T^{0,1}M$. Fibré vectoriel complexe de dimension $n$ dont les sections sont appelés \textit{champs de vecteurs \mbox{($I$-)antiholomorphes}}.
\item Une opération $X \mapsto \bar{X}$ sur $T_\C M = T_\R M \otimes_\R \C$ qui échange $T^{1,0}M$ et $T^{0,1}M$.
\end{itemize}
\paragraph{} L'espace des $1$-formes réelles $\Omega_{\R,M} := \Hom_\R(T_\R M,\R)$.\\
Si $K$ est un tenseur $T_\R M \rightarrow T_\R M$, alors $K$ agit naturellement sur $\Omega_{\R,M}$ par composition à droite, on le notera toujours $K$.
\paragraph{} Son complexifié $\Omega_{\C,M} := \Hom_\C(T_\C M,\C) = \Hom_\R(T_\R M,\C)$. On a la décomposition obtenue par dualité~:
\[
\Omega_{\C,M} = \Omega^{0,1} \oplus \Omega^{1,0}
\]
\begin{itemize}
\item $\Omega^{1,0} := (T^{0,1}M)^\bot$ faisceau des formes qui s'annulent sur les $(0,1)$-vecteurs. Fibré vectoriel complexe de dimension $n$.\\
C'est également l'espace des formes propres de valeur propre $i$ pour l’opérateur $I$.
\item $\Omega^{0,1} := (T^{1,0}M)^\bot$ faisceau des formes qui s'annulent sur les $(1,0)$-vecteurs. Fibré vectoriel complexe de dimension $n$.\\
C'est également l'espace des formes propres de valeur propre $-i$ pour l'opérateur $I$.
\end{itemize}
\paragraph{} On définit les $m$-formes à valeur complexes par~:
\[
\Omega^m := \bigwedge^m \Omega_{\C,M}
\]
C'est le faisceau des formes $m$-linéaires alternées sur $TM$ à valeurs complexes. On remarquera que $\Omega^0 = \Cc^\infty_M$ faisceau des fonctions complexes lisses.
\paragraph{ \label{forme_type}} Enfin, on définit les $(p,q)$-formes de la façon suivante~:
\[
\Omega^{p,q} : = \bigwedge^p \Omega^{1,0}  \wedge \bigwedge^q \Omega^{0,1} \; \subset \; \Omega^m
\]
Si on pose $m=p+q$. C'est également le faisceau des formes $m$-linéaires alternées sur $T_\C M$ qui s'annulent sur les $m$-uplets de vecteurs $(X_1,\cdots X_{m})$ dès lors que
\begin{itemize}
\item au moins $p+1$ des $X_i$ sont de type $(1,0)$
\item ou au moins $q+1$ des $X_i$ sont de type $(0,1)$.
\end{itemize}
On a alors la décomposition
\[
\Omega^m = \bigoplus_{p+q = m} \Omega^{p,q}
\]
\paragraph{} Les opérateurs $\dd : \Omega^k \rightarrow \Omega^{k+1}$ en définissant $(\dd \theta)(X)$ pour $\theta \in  \Omega^k$, $X=(X_1,X_2, \cdots , X_{k+1})$ où les $X_i \in TM$ par~:
\[
\sum_{1 \leq j \leq k} (−1)^j X_j\left(\theta\left(\check{X}^j\right)\right)+
\sum_{1 \leq j < i \leq k} (−1)^{j+i} \theta\left([X_j,X_i ], \check{\check{X}}^{j,i} \right)
\]
\begin{itemize}
\item Dans le cas $k=0$, la deuxième partie de la formule est vide, et on retrouve l’opération $f \mapsto X(f)$, ainsi $(\dd f)(X) = X(f)$.
\item Cette définition intrinsèque coïncide dans des coordonnées avec
\[
\dd \theta = \dd \left(\theta_K \dd x^K\right) = \dpp{\theta_K}{x^i}\dd x^i \wedge \dd x^K
\]
\item L’opérateur $d$ sur $ \Omega^k$ satisfait la règle de \textsc{Leibnitz} : 
\[
\dd\,(\alpha \wedge \beta) = \dd \alpha \wedge \beta +(-1)^{\deg(\alpha)} \alpha \wedge \dd \beta
\]
\end{itemize}
\paragraph{} On dispose naturellement des projections
\[
\pi^{p,q} : \Omega^{p+q} \longrightarrow \Omega^{p,q}
\]
On peut dès lors définir les opérateurs $\partial$ et $\dbarre$ comme
\begin{itemize}
\item la partie de type $(p+1,q)$ de la différentielle d'une $(p,q)$-forme : $\partial = \pi^{p+1,q} \circ \dd_{\,\vert \Omega^{p,q}}$
\item la partie de type $(p,q+1)$ de la différentielle d'une $(p,q)$-forme : $\dbarre = \pi^{p,q+1} \circ \dd_{\,\vert \Omega^{p,q}}$
\end{itemize}
\paragraph{} On définit $\Oo_M$ (ou $\Oo_{(M,I)}$ si il y a ambiguïté) le faisceau des fonctions holomorphes sur $M$ à valeurs complexes comme le noyau de l’opérateur $\dbarre : \Cc^\infty_M \rightarrow \Omega^1$. C'est automatiquement un sous-faisceau de $\Cc^\omega_M$ (conséquence de la formule de \textsc{Cauchy}).\\
De même on définit $\Oo_M(E)$ pour $E \rightarrow M$ fibré vectoriel holomorphe, comme le faisceau des sections holomorphes de $E$.
\paragraph{\label{integrabilité}} La structure presque complexe $I$ est dite \emph{intégrable} si l'une des conditions équivalentes est satisfaite~:
\begin{enumerate}[($i$)]
\item Le tenseur de \textsc{Nijenhuis} défini par~: $N_I(X,Y) := [ I X , I Y ] - I [ X , I Y ] - I [ I X , Y ] + I^2 [ X , Y ]$ est identiquement nul.
\item Pour tout champ de vecteur complexes $X,Y$ sur $M$ satisfaisant $I X=iX$ et $I Y=iY$, le crochet de \textsc{Lie} $[X,Y]$ satisfait également $I[X,Y] = i[X,Y]$
\item L'espace tangent $I$-holomorphe $TM = T^{1,0}M \subset T_\C$  est stable par crochet de Lie. C'est-à-dire $[TM,TM] \subseteq TM$
\item Pour toute famille $(\omega^\alpha)_\alpha$ de $(1,0)_I$-formes sur $M$ de rang $n$, et pour tout $\alpha$, $\dd \omega^\alpha = \theta^\alpha_\beta \wedge \omega^\beta$ pour des $1$-formes $\theta^\alpha_\beta$.
\item Le dual de l'espace tangent $I$-holomorphe $\Omega^{1,0} \subseteq \Omega^1$ satisfait $\dd \Omega^{1,0} \subseteq \Omega^1 \wedge \Omega^{1,0}_M$.
\item $\dd = \partial + \dbarre$, ce qui signifie que la différentielle d'une $(p,q)$-forme est une somme de $(p+1,q)$ et $(p,q+1)$-formes.
\item $\dd  = \partial + \dbarre$ sur $\Omega^1_M$.
\item Le diagramme suivant commute~:
\begin{center}\begin{tikzpicture}
\matrix (m) [matrix of math nodes, row sep=2em,
column sep=2.5em, text height=1.5ex, text depth=0.25ex]
{ \Omega^{0,1} & \Omega^1 & \Omega^{1,0} \\
  \Omega^{0,2} & \Omega^2 & \Omega^{2,0} \\ };
\path[->, font=\scriptsize]
(m-1-2) edge node[auto] {$\pi^{0,1}$} (m-1-1)
(m-1-2) edge node[auto] {$\pi^{1,0}$} (m-1-3)
(m-2-2) edge node[auto] {$\pi^{0,2}$} (m-2-1)
(m-2-2) edge node[auto] {$\pi^{2,0}$} (m-2-3)
(m-1-1) edge node[auto] {$\dbarre$} (m-2-1)
(m-1-2) edge node[auto] {$d$} (m-2-2)
(m-1-3) edge node[auto] {$\partial$} (m-2-3);
\end{tikzpicture}\end{center}
Cette propriété d'intégrabilité est détaillé en appendice \autoref{preuveintegrabilite}.
\item $\partial^2 f = 0$ pour tout $f \in\Gamma(M,\Cc^\infty)$.
\end{enumerate}

\paragraph*{\todo}
\begin{itemize}
\item Anti-symétrisation
\item Notation tensorielle
\item $\dd$ (définition tensorielle)
\end{itemize}

\paragraph*{Formulaire.}
\begin{align*}
\dd f & = \dpp{f}{z}\ \dd z + \dpp{f}{\bar{z}}\ \overline{\dd z} \\
\dbarre f & = \dpp{f}{z}\ \dbarre z + \dpp{f}{\bar{z}}\ \overline{\partial z} \\
\partial f & = \dpp{f}{z}\ \partial z + \dpp{f}{\bar{z}}\ \overline{\dbarre z}
\end{align*}


\section{Espace des twisteurs d'une variété hyperkahlérienne}
\subsection{Notations}
\cite{Hitchin-Karlhede} \paragraph*{}Le but de cette partie est d'appliquer les résultats précédents à $M=Z :=X \times \Pro^1$ l'espace des twisteurs d'une variété hyperkählérienne $(X,(I,J,K),g)$ que l'on supposera construite à partir d'une variété symplectique holomorphe $(X,I,\sigma)$ et de la donnée d'une classe de Kähler $\omega_I$ qui détermine d'après le théorème de Yau une unique métrique Ricci-plate $g$. 

\paragraph*{}$Z$ est de dimension réelle $\dim_\R(X)+2 = 4n+2$. On notera $N = 2n+1$, ainsi $\dim_\R(Z) = 2N$.

On utilisera les indices suivants~:
\begin{itemize}
\item Les indices $\alpha,\beta,\gamma,\delta$ vont varier entre $1$ et $2n$ (soit $2n$ indices). Ils correspondront aux tenseurs sur $X$.
\item Les indices $a,b,c,d$ vont varier entre $0$ (sur $\Pro^1$) et $2n$ (soit $N$ indices). Ils correspondront aux tenseurs sur $Z$.
\item Les indices $p,q,r,s$ vont varier entre $1$ et $4n+1$ en évitant $2n+1$ (soit $4n$ indices). Ils correspondront aux tenseurs réels sur $X$. Tout indice $p$ est de la forme $\alpha$ ou $\alpha+N$.
\item Les indices $i,j,k,l$ vont varier entre $0$ et $4n+1$ (soit $2N$ indices). Ils correspondront aux tenseurs réels sur $Z$. Les indices $i=0$ et $i=N$ sont particuliers et correspondent aux tenseurs provenant de $\Pro^1$. Tout indice $i$ est de la forme $a$ ou $a+N$.
\end{itemize}

\paragraph*{Structure presque-complexe.}La structure presque-complexe $\mathbb{I}$ sur $Z = X \times \Pro^1$ est donnée par~:
\[
\mathbb{I} = \left(
\dfrac{1-\zeta \bar{\zeta}}{1+\zeta \bar{\zeta}}I + \dfrac{\zeta + \bar{\zeta}}{1+\zeta \bar{\zeta}}J + \dfrac{i(\zeta - \bar{\zeta})}{1+\zeta \bar{\zeta}}K \; , \; I_0
\right)
\]



\subsection{Les structures complexes $J$ et $K$ sur $X$}
Les trois structures complexes $I,J,K$ d'une variété hyperkählérienne pointée, doivent satisfaire les relations "quaternioniques". En particulier $K = IJ$ et $IJ = -JI$.

On cherche donc sur $(X,I)$ une structure complexe $J$ satisfaisant la dernière relation.

Sur un ouvert $U$ de $X$ on a des coordonnées complexes $u^\tau$ qui se décomposent en $4n$ coordonnées réelles $x^p$ (par exemple $u^\tau = x^\tau + i x^{\tau+2n}$) qui induisent une trivialisation locale du fibré tangent \[
T(X_\text{diff})_{\vert U} \cong \bigoplus_p \R \dpp{}{x^p}
\]
Dans cette base la structure $I$ est donnée par la matrice
\[
\begin{pmatrix}
0 & -\Id_{2n} \\ 
\Id_{2n} & 0
\end{pmatrix} 
\]
On va chercher $J$ parmi les matrices qui vérifient $IJ=-JI$ et qui satisfont
\begin{equation}\label{symplecticJ}
\sigma(X,Y) = g(JX,Y) + ig(IJX,Y) = g^\C((1+iI)JX,Y)
\end{equation}
où $\sigma$ est la forme symplectique sur $(X,I)$ et $g$ est la métrique Ricci-plate obtenue par le théorème de \textsc{Yau}.

Posons $A = J + iK$ \label{A}, alors l'équation précédente se réécrit
\[
\sigma(X,Y) = g(AX,Y)
\]
Ce qui donne en notation tensorielle
\[
\sigma_{pq} = g_{rq}A^r_p
\]
On peut dès lors inverser $g$ pour obtenir $A_p^q = \sigma_{pr}g^{rq}$. On peut montrer facilement les propriétés suivantes vérifiées par le tenseur $A$~:
\begin{enumerate}[(\itshape i\,\upshape )]
\item $\Re(A) = J$ et $\Im(A) = K$.\label{eqJ}
\item $A = (1 + iI)J$ et donc $(1-iI)A = 0$. \label{Aproj}
\item $A^2  = 0$
\item $AI =  -IA$
\item $A \in \Oo(g) \otimes \C $\label{Aorthoplex} \marginpar{interpréter \ref{Aorthoplex}}
\item $\nabla A = 0$.
\end{enumerate}
Le tenseur $A$ est parallèle car $J$ et $K$ le sont. Ce qui s'écrit en notation tensorielle
\[
\dpp{A^p_q}{x^r} + \Gamma^p_{sr} A^s_q = 0
\]
où les $\Gamma^p_{qr}$ sont les symboles de \textsc{Christoffel} associés à la métrique $g$.

\subsubsection{Projections et espaces propres holomorphes}
On notera par la suite $P$ ou $P(\zeta)$ si le contexte l'exige, la projection sur l'espace propre $I_\zeta$-holomorphe
\begin{equation}
P(\zeta) = \demi \left( 1 - i I_\zeta \right) : T^\C M \rightarrow T^\C M
\end{equation} 
Le projecteur associé sur l'espace propre antiholomorphe est $\bar{P}$.

Il est à noter que $P(0) = \demi(1-iI)$ et $P(1) = \demi(1-iJ)$.

De plus on a la propriété d'adjonction suivante qui résulte de l'orthogonalité des structures $I_\zeta$ par rapport à $g$:
\begin{equation}\label{Padjoint}
\forall X,Y \quad g(PX,Y ) = g(X,\bar{P}Y)
\end{equation}

\paragraph*{Remarque : Critère métrique d'intégrabilité}
L'intégrabilité de la structure $I_\zeta$ équivaut à
\begin{equation}
\forall X , \forall Y, \forall Z , \quad  g([PX,PY],PZ) = 0
\end{equation}

\subsection{Structure presque-complexe sur l'espace des twisteurs}
En regroupant ce qu'on a obtenu, on peut écrire $\mathbb{I}$ de la façon suivante~:
\[
\mathbb{I}_i^j = \left\lbrace
\begin{array}{cr}
\dfrac{1-\zeta \bar{\zeta}}{1+\zeta \bar{\zeta}} \ I_i^j + \dfrac{\zeta + \bar{\zeta}}{1+\zeta \bar{\zeta}}\ \Re(A_i^j) + i\dfrac{\zeta - \bar{\zeta}}{1+\zeta \bar{\zeta}}\ \Im(A_i^j)  & i,j \notin \{0,N\}\\ 
1 & (i,j) = (0,N)\\
-1 & (i,j) = (N,0)\\
0 & \text{sinon}
\end{array}\right.
\]
où $\zeta = x^0+ix^N$.
Donc la structure complexe $\mathbb{I}$ est la partie réelle du tenseur suivant que l'on notera $\Upsilon$ avec
\[
\Upsilon_p^q = \dfrac{1-\zeta \bar{\zeta}}{1+\zeta \bar{\zeta}}\ I_p^q + \dfrac{2\zeta}{1+\zeta \bar{\zeta}}\ A_p^q
\]

\subsection{$1$-formes holomorphes}
\cite{Hitchin-Karlhede} On a l'application suivante
\begin{equation}
\begin{pmatrix}
\Omega^1_M & \longrightarrow & \Omega^1_M \\ 
\theta &  \mapsto &  (1 + \zeta K) \theta
\end{pmatrix} 
\end{equation}
qui envoie $\Omega^{1,0}_0$ sur $\Omega^{1,0}_\zeta$, c'est-à-dire qui envoie les formes $I$-holomorphes sur des formes $I_\zeta$-holomorphes. L'application induite entre ces deux espaces est de plus bijective pour tout $\zeta$ (le cas $\zeta = \pm i$ est géré en remarquant que $\theta = P(0)^*\theta$ et que le rang de $P(0)(1+\zeta  K)$ ne dépend pas de $\zeta$).

\appendix

\bibliographystyle{amsalpha}
\bibliography{Demailly.bib}

\end{document}
