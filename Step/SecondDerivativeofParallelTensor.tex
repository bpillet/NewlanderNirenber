\documentclass[11pt,makeidx]{amsart}%		twocolumn
\usepackage[couleur]{amsdip}%								ct, thm, couleur, draft, minimal

\selectlanguage{english}
%\geometry{a4paper, hmargin=.07 \paperwidth, vmargin=.10 \paperheight}
%\geometry{a4paper, hmargin=0.18 \paperwidth, vmargin=0.18 \paperheight}

%\setcounter{secnumdepth}{5}
\renewcommand{\thesubsection}{\arabic{subsection}}
\numberwithin{equation}{subsection}
%\renewcommand{\theequation}{\thesubsection.\arabic{equation}}

\renewcommand{\Re}{\texttt{Re}}
\renewcommand{\Im}{\texttt{Im}}

\title{}

\begin{document}
\section{}
\subsection{Context} Let $J$ be a complex structure on $X$ satisfying $IJ=-JI$ and such that $(X,g,J)$ is Kähler. Therefore 
\begin{equation}\label{parallel}
\nabla J = 0
\end{equation}
and moreover one can take coordinates $x^a$ on $X$ in a neighboorhood of $O$ such that
\begin{equation}\label{semi-normal}
\forall a, b \quad 
\nabla_{\partial_a} \partial_b = 0
\end{equation}
at the point $O$. Such coordinate need not be normal at $O$ ($g_{ij} \neq \delta_{ij}$).
\subsection{Vanishing of the first derivative} In such coordinate, \eqref{parallel} readily gives
\begin{equation}\label{first_derivative}
\partial_a J^j_i |_O = 0
\end{equation}
Indeed, \eqref{semi-normal} implies the vanishing of the Christoffel symbols $\Gamma^c_{ab}$ at $O$ and parallelism given by \eqref{parallel} reads
\begin{equation}
\partial_a J^j_i = - \Gamma^j_{ak} J^k_i + \Gamma^k_{ai} J^j_k
\end{equation}
which therefore goes to $0$ at $O$.
\subsection{Second derivative of the tensor $J$} One wants to compute higher derivative of the tensor $J$ at $O$
\begin{equation}\label{second_derivative}
\partial_b  \partial_a J_i^j \vert_O
\end{equation}
Take $A = \partial_a$, $B= \partial_b$, $X=\partial_i$, $Y = \partial_j$, then
\begin{equation}
\partial_b \partial_a \left( J^k_i g_{kj} \right) = B\left( A \left( g\left( JX, Y \right) \right) \right)
\end{equation}
The right-hand side can be expanded as follows
\begin{align*}
\partial_b \partial_a \left( J^k_i g_{kj} \right) &= B\left( A \left( g\left( JX, Y \right) \right) \right) \\
&= B\left( g\left( \nabla_A(JX), Y \right) +  g\left(JX, \nabla_AY \right)\right) \\
&= g\left( \nabla_B\nabla_A(JX), Y \right) 
	+ g\left( \nabla_A(JX), \nabla_BY \right) 
	+  g\left(\nabla_B(JX), \nabla_AY \right)
	+  g\left(JX, \nabla_B\nabla_AY \right) \\
\end{align*}
by using the parallelism of the metric $g$ under the Levi-Civita connection $\nabla$.

Moreover \eqref{parallel} gives for all vector $C$ and vector field $Z$,
\begin{equation}\label{nablaJZ}
\nabla_C (J Z ) = J (\nabla_C Z)
\end{equation}

Combining the equations above with \eqref{semi-normal} yields
\begin{equation}\label{righthandside}
\partial_b \partial_a \left( J^k_i g_{kj} \right) \vert_O = g\left( J\nabla_B\nabla_A X, Y \right) +  g\left(JX, \nabla_B\nabla_AY \right)
\end{equation}

On the other hand, 
\begin{equation}
\partial_b \partial_a \left( J^k_i g_{kj} \right) = 
g_{kj} \left(\partial_b \partial_a J^k_i\right)
+ \left( \partial_a J^k_i \right) \left(\partial_b g_{kj} \right)
+ \left( \partial_b J^k_i \right) \left(\partial_a g_{kj} \right)
+ J^k_i \left(\partial_b \partial_a g_{kj}\right)
\end{equation}
As for $x$ goes to $O$, \eqref{first_derivative} yields
\begin{equation}\label{lefthandside}
\partial_b \partial_a \left( J^k_i g_{kj} \right)\vert_O = 
g_{kj} \left(\partial_b \partial_a J^k_i\right)\vert_O 
+ J^k_i \left(\partial_b \partial_a g_{kj}\right)\vert_O
\end{equation}

\subsection{The computation of $\nabla_B \nabla_A X$}
In all generality, one has
\begin{equation}\label{Christoffel}
\nabla_C Z = C^a \left(\nabla_a X\right)^k = C^a\left(\partial_a Z^k + \Gamma^k_{al} Z^l \right)\partial_k
\end{equation}
Hence, with $C=B= \partial_b$
\begin{equation}
\nabla_B Z = \left(\partial_b Z^k + \Gamma^k_{bl} Z^l \right)\partial_k
\end{equation}
Applying this to $Z = \nabla_A X=\left(\partial_a \delta^k_i + \Gamma^k_{al} \delta^l_i \right)\partial_k = \Gamma^k_{ai}\partial_k$, one finally obtains
\begin{equation}
\nabla_B \nabla_A X = \left( \partial_b \Gamma^k_{ai} + \Gamma^k_{bl}\Gamma^l_{ai} \right) \partial_k
\end{equation}
And moreover evaluating in $O$ kills the Christoffel symbols and gives
\begin{equation}\label{nablaBAX}
\nabla_B \nabla_A X \vert_O = \partial_b \Gamma^k_{ai} \partial_k
\end{equation}

And similarly
\begin{equation}\label{nablaBAY}
\nabla_B \nabla_A Y \vert_O = \partial_b \Gamma^k_{aj} \partial_k
\end{equation}

\subsection{Conclusion and links with curvature}
In this paragraph, all derivative will be implicitly evaluated at $O$.

First by pluggin \eqref{nablaBAX} and \eqref{nablaBAY} in \eqref{righthandside}, one gets
\begin{equation}\label{righthandside2}
J^l_k g_{lj} \partial_b \Gamma^k_{ai} + J^l_i g_{kl}\partial_b \Gamma^k_{aj} 
\end{equation}
And therefore combininig \eqref{righthandside2} and \eqref{lefthandside} one gets
\begin{equation}
g_{kj} \left(\partial_b \partial_a J^k_i\right)
+ J^k_i \left(\partial_b \partial_a g_{kj}\right) = J^l_k g_{lj} \partial_b \Gamma^k_{ai} + J^l_i g_{kl}\partial_b \Gamma^k_{aj} 
\end{equation}
which yields
\begin{equation}\label{final}
\partial_b  \partial_a J_i^j = J^j_k \partial_b \Gamma^k_{ai} +g^{jm} g_{kl}J^l_i \partial_b \Gamma^k_{am} - g^{jm}J^k_i \left(\partial_b \partial_a g_{km}\right)
\end{equation}

The Riemann curvature tensor express in $O$ in the coordinates $x^a$
\begin{equation}
R^k_{mab} = \partial_a \Gamma^k_{bm} - \partial_b \Gamma^k_{am}
\end{equation}
keeping in mind the symmetry of the Christoffel symbols, namely
\begin{equation}
\Gamma^k_{ij} = \Gamma^k_{ji}
\end{equation}
And
\begin{equation}
\partial_b \partial_a g_{km} =  -\dfrac{1}{3}\left( 
R_{kbma} - R_{kamb}
\right)
\end{equation}
Hence
\begin{equation}
g^{jm} \left(\partial_b \partial_a g_{km}\right) = -\dfrac{1}{3}\left( 
g^{jm}R_{makb} - g^{jm}R_{mbka}
\right)
=-\dfrac{1}{3}\left( 
R^j_{akb} - R^j_{bka}
\right)
\end{equation}


After some manipulation one gets
\begin{equation}\label{final2}
\partial_b  \partial_a J_i^j = J^j_k \partial_b \Gamma^k_{ai} + J^l_ig_{kl}g^{jm} \partial_b \Gamma^k_{am} +\dfrac{1}{3}\left( 
R^j_{akb} - R^j_{bka}
\right)J^k_i
\end{equation}
which only depends on $J(O)$ and on the curvature and the christoffel symbols of $g$. For instance assuming 
…

\begin{equation}
\partial_b  \partial_a J_i^j = J^l_k \left(
\delta^j_l \partial_b \Gamma^k_{ai} + \delta^k_i \left(
g_{pl}g^{jm} \partial_b \Gamma^p_{am} +\dfrac{1}{3}\left( 
R^j_{alb} - R^j_{bla}
\right)
\right)
\right)
\end{equation}

\appendix
\bigskip
\section{other relations}
Orthogonality of $J$
\begin{equation}
g(JZ,Z') = -g(Z,JZ')
\end{equation}
in coordinates
\begin{equation}
g_{kl}J^l_i = -J^l_k g_{li}
\end{equation}
\end{document}